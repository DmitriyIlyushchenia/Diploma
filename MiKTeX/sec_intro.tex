\sectioncentered*{Введение}
\addcontentsline{toc}{section}{Введение}
\label{sec:intro}


В наши дни рыночные отношения диктуют условия функционирования предприятий. Для полноценного участия в рыночных отношениях даже успешно и полноценно функционирующие предприятия вынуждены прибегать к кредитованию в банках. Банк, как юридическое лицо, имеет право, помимо всего прочего, на размещение привлеченных денежных средств от своего имени и за свой счет на условиях возвратности, платности и срочности.[Банковский кодекс РБ]
Предоставляя ссуду предприятию или физическому лицу, банк идет на некоторый риск, если заниматель в последствии окажется неспособным в срок не только вернуть сумму займа, но и оплатить ссудный процент в соответствии с принципом платности.

Главным способом минимизировать риск является наиболее адекватная оценка возвратности кредита. Оценивание состоит в определении границ финансовой устойчивости предприятия, поскольку недостаточная финансовая устойчивость может привести к неплатёжеспособности и, в конечном счёте, к банкротству.
В отделах анализа банков зачастую заняты крупные штаты сотрудников, которые в течение довольно длительного времени рассматривают каждую заявку, которая включает в себя финансовую отчетность предприятия о своей деятельности за определенный промежуток времени.

В наше время бурного развития информационных технологий данный аспект работы банков всё еще слабо автоматизирован. Причинами этого можно считать, по всей видимости, нежелание рисковать крупными суммами денег, делая попытки вводить элементы автоматизации и интеллектуализации, а также уверенность в исключительной прерогативе человека наиболее полно изучить и составить представление о платёжеспособности заемщика исходя из предоставленной документации

В данном дипломном проекте рассматривается разрабатываемая модель системы, способная провести быстрый и наиболее детальный анализ финансового состояния предприятия, оптимизировать штат сотрудников отделов кредитования банков, а также эффективно работать в условиях отсутствия четких правил формализации при отражении экономических показателей того или иного аспекта деятельности предприятия.

Целью данного проекта является разработка системы, способной значительно ускорить довольно долгосрочные процессы и упростить взаимоотношение кредитополучатель-заёмщик. Для достижения поставленной цели необходимо выполнить следующие задачи: 
\begin{itemize}
\item Разработка алгоритмов преобразования исходной документации к формализованным данным.
\item Создание и систематизация шаблонов входных данных.
\item Программная реализация и внедрение имеющихся алгоритмов финансового анализа.
\item Определение критериев оценивания и выделение основных экономических показателей для сопоставления с этими критериями
\end{itemize}

