\section{Характеристика места практики}
\label{sec:practice:itechart_characteristics}
\newcommand{\company}{\mbox{<<Техартгруп>>}}

Частная компания ООО~\company{} занимается предоставлением услуг по разработке программного обеспечения, консалтингом и внедрением корпоративных решений для многих европейских и североамериканских компаний.
Компания была основана в 2003 году.
Штаб"=квартира компании расположена в городе Изелин (Iselin), штат Нью-Джерси, Соединенные Штаты Америки.
Центры разработки компании расположены в Минске, Беларусь и Киеве, Украина.
Среди партнеров компании можно выделить такие общеизвестные компании как Microsoft, Oracle, IBM, Adobe и другие.

Компания предоставляет следующие услуги своим заказчикам~
\cite{itechart_2013}:
\begin{itemize}
  \item разработка программного обеспечения;
  \item интеграция решений в существующую инфраструктуру;
  \item разработка мобильных приложений;
  \item миграция больших объемов данных;
  \item проведение тестирования и контроля качества;
  \item поддержка существующих решений;
  \item консалтинг в сфере информационных технологий;
  \item администрирование баз данных;
  \item управление инфраструктурой;
  \item управление проектами.
\end{itemize}

Рабочая модель взаимодействия с заказчиками "--- оффшорный центр разработки.
Данная модель представляет собой виртуальную команду разработчиков программного обеспечения.
Команда создается в соответствии с требованиями клиента относительно проекта и специфики его бизнеса и выступает в качестве удаленного расширения внутреннего штата компании клиента.

В соответствии с требованиями клиентов компания предлагает различные модели взаимодействия.
Разработка на стороне клиента "--- данный подход находит свое применение для наиболее сложных и больших проектов, когда нужно тесное взаимодействие заказчика и команд исполнителя, или когда разработка не может быть передана в другое место, например по причине законодательных ограничений, принятых в стране клиента. 
Оффшорная организация "--- вся работа по удовлетворению потребностей заказчика выполняется на стороне компании"=исполнителя, данная модель является наиболее экономичной для заказчика.
При смешанной организации основная работа выполняется на стороне \company{}, но управление проектом и выработка бизнес требований выполняется представителем \company{} на стороне клиента или представителем клиента на стороне компании"=исполнителя.
Данная модель часто используется для больших проектов, когда нужно соблюдать баланс между стоимостью проекта и эффективностью взаимодействия с клиентом~\cite{itechart_2013}.   
В качестве примера компаний"=клиентов \company{} можно привести следующие компании: Coca-Cola, Disney, FedEx, Gain Capital, 10gen и другие общеизвестные компании.

Для более удобного управления структурная организация компании представляет из себя множество отделов.
В компании присутствуют отделы занимающиеся разработкой мобильных приложений для iOS и Android, разработкой приложений для \dotnet{}, разработкой приложений для платформы Java, также есть отделы разработки, специализирующиеся на других технологиях. 
Управление компанией осуществляет административный отдел. 
Также присутствуют отделы материально"=технического обеспечения, тестирования и контроля качества.
Каждый отдел имеет своего руководителя с которым решаются многие вопросы, возникающие у сотрудников отдела.

Кроме деятельности направленной на зарабатывание денег, компания занимается обучением студентов.
В компании с недавнего времени проходят тренинги по веб"=разработке и разработке на платформе \dotnet{}.
По результатам курсов многим студентам предлагают работать в компании.
Компания также сотрудничает и помогает различным университетам в Беларуси.
Благодаря помощи компании был модернизирован студенческий читальный зал~\No{}1 в БГУИР.

В компании работает много молодых и зрелых специалистов.
Компания хорошо относится к своим сотрудникам, созданы условия для отдыха и развлечения сотрудников.
В офисе компании есть комната для отдыха и развлечений, созданная специально для сотрудников.
В теплое время года компания часто организует активный отдых за городом для своих работников.

Прохождение преддипломной практики было в команде занимающейся поддержкой и развитием существующей инфраструктуры для компании GAIN Capital Holdings, Inc.
Данная компания является пионером в области онлайн торговли иностранными валютами и является владельцем бренда FOREX.com.
Результаты, полученные в ходе выполнения индивидуального задания, в данный момент используются компанией GAIN Capital для рассылки уведомлений для своего приложения FOREXTrader for iPhone\texttrademark.
Разработанное решение было гармонично вписано в существующую сервисно"=ориентированную архитектуру инфраструктуры GAIN Capital.